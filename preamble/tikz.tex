% TikZ for Feynman diagrams
%%%%%%%%%%%%%%%%%%%%%%%%%%%%%%%%%%%%%%%
\usetikzlibrary{positioning}
\usetikzlibrary{arrows.meta}
\usetikzlibrary{decorations.markings}
\usetikzlibrary{calc}
   
\tikzset{
   linePlain/.style={draw=black, thick},
   lineWithArrow/.style={draw=black, thick, postaction={decorate},decoration={markings,mark=at position .6 with {\arrow[scale=1.75]{stealth}}}},
   lineWithArrowInline/.style={draw=black, semithick, postaction={decorate},decoration={markings,mark=at position .7 with {\arrow[scale=1.75]{stealth}}}},
   vertex/.style={draw, shape=circle, fill=black, minimum size=1.1mm, inner sep=0mm, outer sep=0mm},
}
\newcommand{\upperloop}[3][]{%
   \draw[#1] let \p1 = ($(#2)-(#3)$) in (#2) arc (0:180:({0.5*veclen(\x1,\y1)});)
} 
\newcommand{\lowerloop}[3][]{%
   \draw[#1] let \p1 = ($(#2)-(#3)$) in (#2) arc (360:180:({0.5*veclen(\x1,\y1)});)
} 
%%%%%%%%%%%%%%%%%%%%%%%%%%%%%%%%%%%%%%%

