% Some useful definitions
%%%%%%%%%%%%%%%%%%%%%%%%%%%%%%%%%%%%%%%
% (some are only included because I needed them for an IOP review)

% Theorem environment (amsthm package)
\newtheoremstyle{customTheorem}% name
	{1ex}% Space above
	{1ex}% Space below
	{}% Body font
	{0pt}% Indent amount
	{\bfseries}% Theorem head font
	{}% Punctuation after theorem head
	{5pt plus 1pt minus 1pt}% Space after theorem head
	{}% Theorem head spec (can be left empty, meaning ‘normal’) 
\theoremstyle{customTheorem}
\newtheorem{theorem}{Theorem}

% Left/right parentheses (mathtools package)
\DeclarePairedDelimiter\paren{\lparen}{\rparen}

% Text in Equations
\newcommand*{\mathtext}[1]{\mathrm{#1}}
\newcommand*{\mathtextit}[1]{\mathit{#1}}

% Exponential and imaginary unit (were required to be upright for IOP)
\newcommand*{\ee}{\mathrm{e}}
\newcommand*{\ii}{\mathrm{i}} 

% Bold italic vector
\newcommand\hmmax{0}%  Patch for bm. With garamondx, the bold italic vector looks better,
\newcommand\bmmax{0}%  it's not great...but ok. Don't know why.
\usepackage{bm}%
\newcommand*{\vect}[1]{\bm{#1}}  

% Unit matrix
\usepackage{dsfont}

% Miscellaneous
\newcommand*{\bigO}{\mathord{\mathrm{O}}}
\newcommand*{\transpose}{\intercal}
\newcommand*{\laplaceOp}{\Delta}
\newcommand*{\discreteLaplaceOp}{\underline{\Delta}}
\newcommand*{\unitMatrix}{\mathds{1}}
\newcommand*{\hermite}{\mathit{He}}
\newcommand*{\legendreTransform}{\mathcal{L}}
\newcommand*{\fourierTransform}{\mathcal{F}}
\newcommand*{\complexPath}{\mathcal{C}}
\newcommand*{\binomCoeff}[2]{\binom{#1}{#2}}
\DeclareMathOperator{\Res}{Res}

% Sets
\newcommand*{\naturals}{\mathbb{N}}
\newcommand*{\integers}{\mathbb{Z}}
\newcommand*{\rationals}{\mathbb{Q}}
\newcommand*{\reals}{\mathbb{R}}
\newcommand*{\complex}{\mathbb{C}}
\newcommand*{\quaternions}{\mathbb{H}}
\newcommand*{\octonions}{\mathbb{O}}
\newcommand*{\lattice}{\mathbb{L}}

% Mathematical functions
\newcommand*{\arcosh}[1]{\mathtext{arcosh}\,#1}
 
% Differentials and integral signs
\newcommand*{\diff}{\mathop{}\!\mathrm{d}}
\usepackage{esint}%  Some more integral signs

% Fock space operations
\newcommand*{\numberOperator}{\mathsf{n}}
\newcommand*{\cre}{\mathsf{c}}
\newcommand*{\ann}{\mathsf{a}}

% Bra-ket notation (little bit optimized for garamondx... not optimal)
\newcommand*{\LLangle}{\langle\kern-2\nulldelimiterspace\langle\kern0.25\nulldelimiterspace}
\newcommand*{\bigLangle}{\big\langle}
\newcommand*{\bigLLangle}{\bigLangle\kern-2.25\nulldelimiterspace\bigLangle}
\newcommand*{\BigLangle}{\Big\langle}
\newcommand*{\BigLLangle}{\BigLangle\kern-2.35\nulldelimiterspace\BigLangle}

\newcommand*{\RRangle}{\rangle\kern-2\nulldelimiterspace\rangle}
\newcommand*{\bigRangle}{\big\rangle}
\newcommand*{\bigRRangle}{\bigRangle\kern-2.25\nulldelimiterspace\bigRangle}
\newcommand*{\BigRangle}{\Big\rangle}
\newcommand*{\BigRRangle}{\BigRangle\kern-2.35\nulldelimiterspace\BigRangle}

\newcommand*{\langleKern}{\langle\kern0.25\nulldelimiterspace}
\newcommand*{\vertLeft}{|\kern0.5\nulldelimiterspace}
\newcommand*{\vertRight}{\kern0.25\nulldelimiterspace|}
\newcommand*{\vertCenter}{\kern0.25\nulldelimiterspace|\kern0.5\nulldelimiterspace}

\newcommand*{\braA}[1]{{\langleKern #1 \vertRight}}
\newcommand*{\ketA}[1]{{\vertLeft #1 \rangle}}
\newcommand*{\braketA}[2]{\langleKern #1 \vertCenter #2 \rangle}

\newcommand*{\braB}[1]{\LLangle #1 \vertRight}
\newcommand*{\ketB}[1]{\vertLeft #1 \RRangle}
\newcommand*{\braketB}[2]{\LLangle #1 \vertCenter #2 \RRangle}

\newcommand*{\braketBA}[2]{\LLangle #1 \vertCenter #2 \rangle}
\newcommand*{\braketAB}[2]{\langleKern #1 \vertCenter #2 \RRangle}
