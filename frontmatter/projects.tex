\chapter{Projects and contributions}

\noindent
The work presented in this thesis is divided into the following three chapters. Each chapter represents a project on which I have worked during my doctoral studies. The projects are summarized on the next pages. \mnote{This page is non-standard. And only looks good if it starts with a \Rmnum{1}.}
%
\begin{enumerate}[
	label=\textbf{\Roman*},
	itemindent=0em, leftmargin=0em, 
	itemsep=0ex, topsep=1ex,
	start=2
]
	\item[{\hyperref[ch:1_chapter]{\textbf{\Rmnum{2}}}}] 
		\textbf{Paralogisms of practical reason}\\
		with \textit{Friedrich Nietzsche.}\\
		%
		Published in~\hyperref[sec:1_chapter_MathAnn]{``The ontological manuals''} (\textit{Math. Ann.} \textbf{534}(4), 432--458 (1870)) to which I contributed as third author.
		%
		\mnote{Random text: }The Ideal can not take account of, so far as I know, our faculties. As we have already seen, the objects in space and time are what first give rise to the never-ending regress in the series of empirical conditions; for these reasons, our a posteriori concepts have nothing to do with the paralogisms of pure reason. As we have already seen, metaphysics, by means of the Ideal, occupies part of the sphere of our experience concerning the existence of the objects in space and time in general.
	%
	\item[{\hyperref[ch:2_chapter]{\textbf{\Rmnum{3}}}}] 
		\textbf{Ampliative judgements}\\
		with \textit{Arthur Schopenhauer.}\\
		Let us suppose that the noumena have nothing to do with necessity, since know- ledge of the Categories is a posteriori. Hume tells us that the transcendental unity of apperception can not take account of the discipline of natural reason, by means of analytic unity. As is proven in the ontological manuals, it is obvious that the transcendental unity of apperception proves the validity of the Antinomies; what we have alone been able to show. 
	%
	\item[{\hyperref[ch:3_chapter]{\textbf{\Rmnum{4}}}}] 
		\textbf{System of transcendental philosophy}\\
		\textit{with Martin Heidegger.}\\
		By virtue of natural reason, our ampliative judgements would thereby be made to contradict, in all theoretical sciences, the pure employment of the discipline of human reason. Because of our necessary ignorance of the conditions, Hume tells us that the transcendental aesthetic constitutes the whole content for, still, the Ideal. By means of analytic unity, our sense perceptions, even as this relates to philosophy, abstract from all content of knowledge.  For these reasons, our a posteriori concepts have nothing to do with the paralogisms of pure reason. 		
\end{enumerate}
%




