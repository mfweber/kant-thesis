\chapter{Introduction}\label{ch:introduction}

	\epigraphhead[0]{\epigraph{\textit{How does your patient, doctor?}\qquad\phantom{}}{--- \textup{Dionysius of Halicarnassus}, \textsc{On Imitation}}}

	\mnote{I found it useful to use the glossaries package for shortcuts to acronyms. See the file ./preamble/glossaries.tex for the following example:
	%
	\begin{itemize}[itemsep=0ex, topsep=1ex]
		\item \acrshort{egt}
		%
		\item \acrlong{egt}
		\item \Acrlong{egt}
		\item \acrlongpl{egt}
		\item \Acrlongpl{egt}
		%
		\item \acrfull{egt}
		\item \Acrfull{egt}
		\item \acrfullpl{egt}
		\item \Acrfullpl{egt}
	\end{itemize}
	%
	The list of acronyms can be printed with the \textbackslash{}printglossary command in main.tex (I didn't print it in my thesis). Hyperlinks and backrefs can be activated in ./preamble/glossaries.tex. See the documentation of the package for further information.} 
	
	\mnote{Random text: }\kant[44]
